\documentclass{article}


% if you need to pass options to natbib, use, e.g.:
%     \PassOptionsToPackage{numbers, compress}{natbib}
% before loading neurips_2023


% ready for submission
\usepackage[preprint]{neurips_2023}


% to compile a preprint version, e.g., for submission to arXiv, add add the
% [preprint] option:
%     \usepackage[preprint]{neurips_2023}


% to compile a camera-ready version, add the [final] option, e.g.:
%     \usepackage[final]{neurips_2023}


% to avoid loading the natbib package, add option nonatbib:
%    \usepackage[nonatbib]{neurips_2023}


\usepackage[utf8]{inputenc} % allow utf-8 input
\usepackage[T1]{fontenc}    % use 8-bit T1 fonts
\usepackage{hyperref}       % hyperlinks
\usepackage{url}            % simple URL typesetting
\usepackage{booktabs}       % professional-quality tables
\usepackage{amsfonts}       % blackboard math symbols
\usepackage{nicefrac}       % compact symbols for 1/2, etc.
\usepackage{microtype}      % microtypography
\usepackage{xcolor}         % colors


\title{Calligraphy: Local LLMs as Writing Assistants}


% The \author macro works with any number of authors. There are two commands
% used to separate the names and addresses of multiple authors: \And and \AND.
%
% Using \And between authors leaves it to LaTeX to determine where to break the
% lines. Using \AND forces a line break at that point. So, if LaTeX puts 3 of 4
% authors names on the first line, and the last on the second line, try using
% \AND instead of \And before the third author name.


\author{%
  Tanay Biradar\\
  UC Santa Barbara\\
  \texttt{tbiradar@ucsb.edu} \\
  \And
  Mateo Wang\\
  UC Santa Barbara\\
  \texttt{mathewwang@ucsb.edu} \\
}


\begin{document}


\maketitle


\begin{abstract}
  In this work, we present "Calligraphy," a system designed to augment the
  capabilities of local Large Language Models (LLMs) as writing assistants.
  Our approach focuses on two main tasks: creative rewording and writing
  critiques. We leverage techniques such as fine-tuning on domain-specific
  data, employing Low-Rank Adapters (LoRA), and post-training quantization to
  enhance the performance of LLMs while maintaining efficiency. Our
  evaluation demonstrates that our system outperforms baseline models in
  terms of perplexity, a measure of language model fluency, and provides
  qualitative improvements in creative writing tasks.
\end{abstract}


\section{Introduction}


The advent of LLMs has revolutionized the field of natural language processing,
providing tools capable of conversational interaction, customization, and
multi-task learning. However, these models are not without their flaws, such as a
propensity for hallucinations, the need for extensive prompt engineering, and
occasional flawed reasoning. Our project, Calligraphy, aims to address these issues
by focuing on enhancing user writing, instead of generating writing for them,
playing up to LLMs' strengths of being able to creatively reword and critique
existing writing, delegating the task of generating substantive, factual content to
the user.


\section{Related works}
\label{rel_works}

The development of our project, Calligraphy, is informed by a rich landscape of
prior work that explores the use of Large Language Models (LLMs) as writing
assistants and creative tools. In this section, we review several key contributions that have influenced our approach.

\subsection{Maggie Appleton's Language Model Sketchbook}

Maggie Appleton's Language Model Sketchbook explores non-chatbot interfaces for
language models, proposing alternative interaction paradigms such as "Daemons" and
"Branches" to assist in writing and reasoning tasks. These concepts emphasize the
role of language models as "epistemic rubber ducks," serving as reflective thinking
partners rather than mere conversational agents. However, this is just a design of
an app and not an implementation. Our project, Calligraphy, actually implements
the idea of leveraging LLMs as reflective writing assistants, focusing on enhancing
writing critiques and creative rewording tasks.

\subsection{DeepL Write}

DeepL Write represents a practical implementation of LLMs in writing assistance,
focusing on aiding users in refining their writing across various languages. It
offers suggestions for grammar and punctuation corrections, tone adjustments, and
creative rephrasing, aiming to enhance clarity, precision, and expressiveness in
written communication. However, its scope is quite limited. Our project,
Calligraphy, aims to build upon this idea by automatically generating writing
critiques as one writes and allowing users to manually pick which word or phrase
they'd like to reword.

\subsection{TextFX}

TextFX is an AI experiment that utilizes Google's PaLM 2 large language model to assist in creative writing tasks[5][8]. It provides a suite of tools for generating literary devices such as synonyms, similes, metaphors, and alliterations, which are similar to the creative rewording tasks our project focuses on.

\subsection{NotebookLM}

NotebookLM is a project that aims to integrate language models into research and non-fiction writing processes[3]. It offers functionalities such as rephrasing sentences, offering critiques, finding evidence for claims, and generating research questions, aligning with our project's goal of enhancing writing critiques.

\subsection{The Inquisitive Code Editor}

The Inquisitive Code Editor is a concept that leverages language models to assist in the coding process by suggesting revisions, fetching evidence, and elaborating on points[3]. Although it is more focused on coding, the underlying idea of using language models as assistants in creative tasks is shared with our project.

\subsection{LangChain}

LangChain is a framework that enables the integration of language models with search APIs to fetch relevant sources and support information retrieval[1]. Our project utilizes LangChain to implement the task of fetching sources in agreement or conflict, which is crucial for our writing critique component.

\subsection{GPT-3.5}

GPT-3.5, developed by OpenAI, is a state-of-the-art language model known for its conversational abilities, natural language understanding, and generation, as well as reasoning capabilities[1]. Our project uses GPT-3.5 to generate data for fine-tuning and as a benchmark for evaluating the performance of our enhanced models.

These related works provide a foundation and context for our project, Calligraphy, which aims to augment local LLMs as writing assistants by focusing on creative rewording and writing critiques. Our approach builds upon and extends the capabilities presented in these projects, contributing to the evolving field of language model applications.

\section{Headings: first level}
\label{headings}


All headings should be lower case (except for first word and proper nouns),
flush left, and bold.


First-level headings should be in 12-point type.


\subsection{Headings: second level}


Second-level headings should be in 10-point type.


\subsubsection{Headings: third level}


Third-level headings should be in 10-point type.


\paragraph{Paragraphs}


There is also a \verb+\paragraph+ command available, which sets the heading in
bold, flush left, and inline with the text, with the heading followed by 1\,em
of space.


\section{Citations, figures, tables, references}
\label{others}


These instructions apply to everyone.


\subsection{Citations within the text}


The \verb+natbib+ package will be loaded for you by default.  Citations may be
author/year or numeric, as long as you maintain internal consistency.  As to the
format of the references themselves, any style is acceptable as long as it is
used consistently.


The documentation for \verb+natbib+ may be found at
\begin{center}
  \url{http://mirrors.ctan.org/macros/latex/contrib/natbib/natnotes.pdf}
\end{center}
Of note is the command \verb+\citet+, which produces citations appropriate for
use in inline text.  For example,
\begin{verbatim}
   \citet{hasselmo} investigated\dots
\end{verbatim}
produces
\begin{quote}
  Hasselmo, et al.\ (1995) investigated\dots
\end{quote}


If you wish to load the \verb+natbib+ package with options, you may add the
following before loading the \verb+neurips_2023+ package:
\begin{verbatim}
   \PassOptionsToPackage{options}{natbib}
\end{verbatim}


If \verb+natbib+ clashes with another package you load, you can add the optional
argument \verb+nonatbib+ when loading the style file:
\begin{verbatim}
   \usepackage[nonatbib]{neurips_2023}
\end{verbatim}


As submission is double blind, refer to your own published work in the third
person. That is, use ``In the previous work of Jones et al.\ [4],'' not ``In our
previous work [4].'' If you cite your other papers that are not widely available
(e.g., a journal paper under review), use anonymous author names in the
citation, e.g., an author of the form ``A.\ Anonymous'' and include a copy of the anonymized paper in the supplementary material.


\subsection{Footnotes}


Footnotes should be used sparingly.  If you do require a footnote, indicate
footnotes with a number\footnote{Sample of the first footnote.} in the
text. Place the footnotes at the bottom of the page on which they appear.
Precede the footnote with a horizontal rule of 2~inches (12~picas).


Note that footnotes are properly typeset \emph{after} punctuation
marks.\footnote{As in this example.}


\subsection{Figures}


\begin{figure}
  \centering
  \fbox{\rule[-.5cm]{0cm}{4cm} \rule[-.5cm]{4cm}{0cm}}
  \caption{Sample figure caption.}
\end{figure}


All artwork must be neat, clean, and legible. Lines should be dark enough for
purposes of reproduction. The figure number and caption always appear after the
figure. Place one line space before the figure caption and one line space after
the figure. The figure caption should be lower case (except for first word and
proper nouns); figures are numbered consecutively.


You may use color figures.  However, it is best for the figure captions and the
paper body to be legible if the paper is printed in either black/white or in
color.


\subsection{Tables}


All tables must be centered, neat, clean and legible.  The table number and
title always appear before the table.  See Table~\ref{sample-table}.


Place one line space before the table title, one line space after the
table title, and one line space after the table. The table title must
be lower case (except for first word and proper nouns); tables are
numbered consecutively.


Note that publication-quality tables \emph{do not contain vertical rules.} We
strongly suggest the use of the \verb+booktabs+ package, which allows for
typesetting high-quality, professional tables:
\begin{center}
  \url{https://www.ctan.org/pkg/booktabs}
\end{center}
This package was used to typeset Table~\ref{sample-table}.


\begin{table}
  \caption{Sample table title}
  \label{sample-table}
  \centering
  \begin{tabular}{lll}
    \toprule
    \multicolumn{2}{c}{Part}                   \\
    \cmidrule(r){1-2}
    Name     & Description     & Size ($\mu$m) \\
    \midrule
    Dendrite & Input terminal  & $\sim$100     \\
    Axon     & Output terminal & $\sim$10      \\
    Soma     & Cell body       & up to $10^6$  \\
    \bottomrule
  \end{tabular}
\end{table}

\subsection{Math}
Note that display math in bare TeX commands will not create correct line numbers for submission. Please use LaTeX (or AMSTeX) commands for unnumbered display math. (You really shouldn't be using \$\$ anyway; see \url{https://tex.stackexchange.com/questions/503/why-is-preferable-to} and \url{https://tex.stackexchange.com/questions/40492/what-are-the-differences-between-align-equation-and-displaymath} for more information.)

\subsection{Final instructions}

Do not change any aspects of the formatting parameters in the style files.  In
particular, do not modify the width or length of the rectangle the text should
fit into, and do not change font sizes (except perhaps in the
\textbf{References} section; see below). Please note that pages should be
numbered.


\section{Preparing PDF files}


Please prepare submission files with paper size ``US Letter,'' and not, for
example, ``A4.''


Fonts were the main cause of problems in the past years. Your PDF file must only
contain Type 1 or Embedded TrueType fonts. Here are a few instructions to
achieve this.


\begin{itemize}


  \item You should directly generate PDF files using \verb+pdflatex+.


  \item You can check which fonts a PDF files uses.  In Acrobat Reader, select the
        menu Files$>$Document Properties$>$Fonts and select Show All Fonts. You can
        also use the program \verb+pdffonts+ which comes with \verb+xpdf+ and is
        available out-of-the-box on most Linux machines.


  \item \verb+xfig+ "patterned" shapes are implemented with bitmap fonts.  Use
        "solid" shapes instead.


  \item The \verb+\bbold+ package almost always uses bitmap fonts.  You should use
        the equivalent AMS Fonts:
        \begin{verbatim}
   \usepackage{amsfonts}
\end{verbatim}
        followed by, e.g., \verb+\mathbb{R}+, \verb+\mathbb{N}+, or \verb+\mathbb{C}+
        for $\mathbb{R}$, $\mathbb{N}$ or $\mathbb{C}$.  You can also use the following
        workaround for reals, natural and complex:
        \begin{verbatim}
   \newcommand{\RR}{I\!\!R} %real numbers
   \newcommand{\Nat}{I\!\!N} %natural numbers
   \newcommand{\CC}{I\!\!\!\!C} %complex numbers
\end{verbatim}
        Note that \verb+amsfonts+ is automatically loaded by the \verb+amssymb+ package.


\end{itemize}


If your file contains type 3 fonts or non embedded TrueType fonts, we will ask
you to fix it.


\subsection{Margins in \LaTeX{}}


Most of the margin problems come from figures positioned by hand using
\verb+\special+ or other commands. We suggest using the command
\verb+\includegraphics+ from the \verb+graphicx+ package. Always specify the
figure width as a multiple of the line width as in the example below:
\begin{verbatim}
   \usepackage[pdftex]{graphicx} ...
   \includegraphics[width=0.8\linewidth]{myfile.pdf}
\end{verbatim}
See Section 4.4 in the graphics bundle documentation
(\url{http://mirrors.ctan.org/macros/latex/required/graphics/grfguide.pdf})


A number of width problems arise when \LaTeX{} cannot properly hyphenate a
line. Please give LaTeX hyphenation hints using the \verb+\-+ command when
necessary.


\begin{ack}
  Use unnumbered first level headings for the acknowledgments. All acknowledgments
  go at the end of the paper before the list of references. Moreover, you are required to declare
  funding (financial activities supporting the submitted work) and competing interests (related financial activities outside the submitted work).
  More information about this disclosure can be found at: \url{https://neurips.cc/Conferences/2023/PaperInformation/FundingDisclosure}.


  Do {\bf not} include this section in the anonymized submission, only in the final paper. You can use the \texttt{ack} environment provided in the style file to autmoatically hide this section in the anonymized submission.
\end{ack}



\section{Supplementary Material}

Authors may wish to optionally include extra information (complete proofs, additional experiments and plots) in the appendix. All such materials should be part of the supplemental material (submitted separately) and should NOT be included in the main submission.


\section*{References}


References follow the acknowledgments in the camera-ready paper. Use unnumbered first-level heading for
the references. Any choice of citation style is acceptable as long as you are
consistent. It is permissible to reduce the font size to \verb+small+ (9 point)
when listing the references.
Note that the Reference section does not count towards the page limit.
\medskip


{
\small


[1] Alexander, J.A.\ \& Mozer, M.C.\ (1995) Template-based algorithms for
connectionist rule extraction. In G.\ Tesauro, D.S.\ Touretzky and T.K.\ Leen
(eds.), {\it Advances in Neural Information Processing Systems 7},
pp.\ 609--616. Cambridge, MA: MIT Press.


  [2] Bower, J.M.\ \& Beeman, D.\ (1995) {\it The Book of GENESIS: Exploring
    Realistic Neural Models with the GEneral NEural SImulation System.}  New York:
TELOS/Springer--Verlag.


[3] Hasselmo, M.E., Schnell, E.\ \& Barkai, E.\ (1995) Dynamics of learning and
recall at excitatory recurrent synapses and cholinergic modulation in rat
hippocampal region CA3. {\it Journal of Neuroscience} {\bf 15}(7):5249-5262.
}

%%%%%%%%%%%%%%%%%%%%%%%%%%%%%%%%%%%%%%%%%%%%%%%%%%%%%%%%%%%%


\end{document}